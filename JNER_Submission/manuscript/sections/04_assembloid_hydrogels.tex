\section*{4. Stem-cell–derived assembloids and hydrogel scaffolds for a living cochlear implant}

\noindent
Biohybrid cochlear implant concepts rely on living neural elements that can mature, respond to therapy, and form or re-form synapses with host targets. In practice this means (i) supplying the \emph{right cells} in the \emph{right microenvironment} and (ii) coupling them to the device through a mechanically and electrochemically compatible interface. In this section we outline why three-dimensional (3D) stem-cell constructs---from simple spheroids to complex \emph{assembloids}---are preferred over dissociated cultures, and how xeno-free hydrogels can provide a cochlea-ready niche that also supports localized delivery and chronic electrical interfacing.

\subsection*{4.1 Why 3D? From spheroids to organoids to assembloids}
Dissociated, two-dimensional SGN cultures show fragile survival and limited maturation, whereas 3D environments provide a pro-survival, pro-maturation niche with improved neuritogenesis, synaptic puncta, and electrophysiological properties closer to native SGNs.\citep{Zine2021StemCells,Sun2023CellProlif,Koehler2017NatBiotech} Spheroids (single- or few-lineage aggregates) are easy to form and improve retention and trophic support after transplantation relative to single-cell suspensions.\citep{Chang2020ActaBiomaterialia} Inner-ear \emph{organoids} add rudimentary cyto-architecture and hair-cell--to--SGN ribbon synapses.\citep{Koehler2013Nature,Koehler2017NatBiotech,Sun2023CellProlif} 

\emph{Assembloids} intentionally combine multiple, precisely chosen cell lineages to recreate elements of the SGN microenvironment---for example SGNs with Schwann cells, satellite glia, and a microvascular compartment---achieving myelination, robust trophic support, and more mature firing phenotypes than neuron-only constructs.\citep{Xia2023StemCellReports,Oliveira2023FrontiersPN} Where hair cells are not required for the intended biohybrid function, assembloids can omit them to reduce complexity while retaining essential SC/vascular support. Emerging small-molecule maturation regimens (e.g., “GENtoniK”) can further accelerate human neuron maturation in 3D systems; when used, these require pharmacological validation of synaptic readouts and careful dose scheduling to maintain viability and lineage identity.\citep{Hergenreder2024NatBiotech}

\subsection*{4.2 Design of SGN assembloids: cell composition and roles}
A practical SGN assembloid includes five components whose functions are complementary: (i) \textbf{SGNs} (afferent neurons, the principal excitable elements); (ii) \textbf{Schwann cells} (myelination distal to the habenula; metabolic and trophic support; debris clearance; immune orchestration);\citep{Oliveira2023FrontiersPN,Moss2024iScience} (iii) \textbf{satellite glia} (ion and neurotransmitter homeostasis around somata); (iv) \textbf{microvascular cells} (endothelial cells and pericytes to resist diffusion limits and enable sustained trophic flux); and (v) \textbf{fibroblast/perineurial cells} (ECM deposition and compartmental definition). Co-culture in 3D favors compact myelin formation around peripheral axons and higher-amplitude/phase-locked firing, with evidence of ribbon synapses in innervated cochlear organoid systems.\citep{Xia2023StemCellReports}

\subsection*{4.3 Hydrogel scaffolds as the SGN niche}
The cochlea imposes stringent constraints on materials: (i) mechanical compliance in the low-kPa regime to avoid disturbing micromechanics; (ii) permeability for oxygen, ions, and macromolecules; (iii) xeno-free chemistry for translation; (iv) acoustic compatibility; and (v) processability for minimally invasive delivery. Several animal-free hydrogels meet these needs to different degrees:

\begin{itemize}
	\item \textbf{VitroGel\textsuperscript{\textregistered} NEURON}: shear-thinning, self-healing polysaccharide network with customizable peptide motifs; $E \approx 0.5$--$2$ kPa; rapid recovery; excellent for low-attenuation acoustic environments.\citep{Zine2021StemCells}
	\item \textbf{PeptiGel\textsuperscript{\textregistered} $\alpha 4$}: self-assembling $\beta$-sheet peptide with intrinsic RGD/IKVAV motifs; transparent; GMP-compatible; $E \approx 1$--$3$ kPa.\citep{Millesi2023ACSAMI}
	\item \textbf{PuraMatrix\textsuperscript{\textregistered} (RADA16)}: very soft nanofibrillar network ($E \approx 0.1$--$1$ kPa); highly hydrated; may micro-collapse without reinforcement.\citep{Zhong2010NNR}
	\item \textbf{RGD–alginate}: enzymatically degradable, Ca$^{2+}$-crosslinked hydrogel; tunable but can gel rapidly and reach higher stiffness ($3$--$20$ kPa) unless crosslinking is carefully limited.\citep{Mooney2011TEA}
	\item \textbf{HyStem-HP}: thiolated hyaluronic acid + gelatin + PEGDA; heparin-binding for sustained neurotrophin delivery; workable but short handling window ($\sim$5 min).\citep{Zhong2010NNR}
	\item \textbf{PEG–DA (photopolymerized)}: highly tunable elastic modulus ($0.5$--$20$ kPa); chemically inert and acoustically acceptable only at low crosslink density; requires RGD/ECM grafting for SGN adhesion.\citep{Mooney2011TEA}
\end{itemize}

A minimal design rule emerges: choose a \emph{xeno-free, soft (0.5--3 kPa), peptide-adhesive} gel that can be injected and quickly recover, then decorate with ECM motifs (RGD/IKVAV/YIGSR) and gradients of neurotrophins (e.g., BDNF/NT-3) to bias neurite trajectories. In several cochlear and neural contexts, peptide-functionalized hydrogels attract SGN neurites and support long-term survival; in inner-ear models, nanofibrillar cellulose and related systems have been used to create a 3D niche with \emph{sustained} factor release.\citep{Pancratov2017ColSurfB,Chang2020ActaBiomaterialia}

\subsection*{4.4 Coupling assembloids to the implant: conductive and anti-fouling interfaces}
Hydrogels can be rendered conductive or used as drug-eluting sleeves on perimodiolar arrays to narrow the electrode--neuron gap while improving chronic electrochemistry.\citep{Dalrymple2020JNE,Chikar2012Biomaterials} Composite gels (e.g., GelMA/PEG blended with PEDOT:PSS) can enhance charge transfer and, in vitro, protect cochlear epithelia from oxidative stress.\citep{Tan2024Biomolecules} In parallel, photografted \emph{zwitterionic} hydrogel coatings on commercial CI materials reduce the foreign body response and fibrosis in vivo, improving the long-term interface for both recording and stimulation.\citep{Horne2023ActaBiomaterialia} 

\subsection*{4.5 Integration with modiolar microchannels and localized delivery}
When positioned along the medial face of a perimodiolar array, an assembloid-bearing hydrogel sleeve can interface with the modiolar surface and potentially leverage canaliculi perforantes (CPS) as micro-conduits for neurite entry and factor exchange (see Sec.~3). A heparinized or nanoporous formulation enables \emph{slow release} of BDNF/NT-3 analogues or RNA cargo at the scala–modiolus boundary, sustaining trophic gradients without repeated infusions.\citep{Chikar2012Biomaterials,Johansen2018PLoSOne,StPeter2022FrontBioeng} Micro-topography on the adjacent polymer shank can further bias growth cone trajectories to the intended stimulation sites, allowing chemical and physical guidance to act in concert.\citep{Truong2021HearRes,Vecchi2024JNE}

\subsection*{4.6 Readouts and benchmarks}
For assembloid–hydrogel systems, we recommend separating \emph{proof of integration} from \emph{performance} claims:
\begin{enumerate}
	\item \textbf{Survival and maturation:} viability and myelination $\geq$8~weeks; expression of SGN subtype markers; firing rate stability under repeated stimulation.\citep{Xia2023StemCellReports,Moss2024iScience}
	\item \textbf{Synaptic measures:} ribbon synapse counts and synaptic protein puncta; pharmacological validation of spontaneous postsynaptic events in 3D cultures.\citep{Hergenreder2024NatBiotech}
	\item \textbf{Guidance efficacy:} fraction of neurites traversing toward the modiolar side and into CPS-accessible zones under defined gradients/topographies.\citep{Truong2021HearRes,Vecchi2024JNE}
	\item \textbf{Electrode coupling:} impedance/phase angle and charge storage capacity with conductive gel coatings versus bare contacts under accelerated soak and stimulation.
	\item \textbf{Tissue response:} fibrosis thickness and macrophage phenotype around coated versus uncoated leads in relevant animal models.\citep{Horne2023ActaBiomaterialia,Fibranz2025JFB}
\end{enumerate}

\subsection*{4.7 Translational notes}
To preserve clinical workflow, the construct should be injectable, self-healing, and visible (e.g., via MR-compatible contrast or OCT) during and after insertion. For regulatory alignment, choose xeno-free chemistries with available GMP lots; minimize persistent animal proteins (e.g., Matrigel, gelatin) unless justified by benefit–risk; and document sterilization compatibility and shelf stability. Acoustic/mechanical testing in saline at body temperature is essential to ensure that the hydrogel sleeve does not degrade micromechanics or increase attenuation in the basal turn.

\medskip
\noindent\textbf{Summary.} Human stem-cell–derived SGN assembloids offer a controllable, multi-lineage route to a living CI interface, while modern, xeno-free hydrogels supply the compliant, bioactive niche and delivery vehicle those cells need. When combined with conductive and anti-fouling coatings, these materials can narrow the effective electrode–neuron distance and maintain implant performance while enabling regeneration-driven improvements over time.
