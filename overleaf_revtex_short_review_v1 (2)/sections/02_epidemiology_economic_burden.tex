\section*{2. Epidemiology and Economic Burden of Hearing Loss}
Hearing loss is a major public health concern that affects everyday communication, education, employment, and healthy aging. In the United States, approximately one in eight individuals (13\%, or about 30 million people) aged 12 years and older has measurable hearing loss in both ears, and 15\% of adults (37.5 million) report at least some trouble hearing.\cite{nidcd2021, cdc2010, cdc2021, wilson2014} These figures meet or exceed the prevalence of other common chronic conditions and emphasize the scale at which hearing impairment consumes health system resources and erodes quality of life.

\subsection*{2.1 Prevalence and risk}
Prevalence rises steeply with age and is compounded by cumulative noise exposure and genetic factors. Globally, disabling hearing loss affects a substantial fraction of older adults, with roughly one quarter of those over 60 years affected, while more than one billion young people are estimated to be at risk of noise-induced, permanent hearing loss.\cite{WHO2025} These epidemiologic trends foreshadow increasing demand for accessible hearing care across the life course.

\subsection*{2.2 Direct and indirect economic costs}
The economic burden of hearing loss spans direct medical spending and substantial indirect costs. Direct costs include clinical evaluations, hearing aids (commonly \$1{,}000--\$4{,}000 per pair), cochlear implantation (\$30{,}000--\$100{,}000 per patient), audiologic rehabilitation, speech--language therapy, and ongoing device maintenance. Indirect costs include reduced productivity and employment: adults with hearing loss have higher odds of unemployment (odds ratio \(\approx 1.98\)), and individual annual income losses up to \$30{,}000 have been reported, aggregating to an estimated \$176 billion per year in the U.S.\cite{SocietyCosts2000, Kim2020, Colburn2019, WHO2025} At the global level, the annual societal cost of unaddressed hearing loss was estimated at \$981 billion in 2020, with quality-of-life losses accounting for about 47\% and a majority of costs (\(\sim\)57\%) incurred outside high-income countries.\cite{McDaid2021}

Strategic investment in hearing care is projected to yield substantial returns. The World Health Organization estimates that scaling essential ear and hearing care interventions to 90\% coverage over the next decade would require an additional \$238.8 billion and generate more than \$2 trillion in productivity gains by 2030.\cite{Tordrup2022} From an individual perspective, early cochlear implantation in children can be cost-saving over the life course: one analysis estimated lifetime healthcare costs of \$489{,}274 for a person born with severe-to-profound loss, reduced to \$390{,}931 (95\% CI \$311{,}976--\$471{,}475) when a cochlear implant is provided before 18 months of age, yielding net lifetime savings of \$98{,}343.\cite{Cejas2024}

\subsection*{2.3 Market context and barriers}
The global cochlear implant market was valued at approximately \$2.42 billion in 2023 and is projected to grow to \$6.63 billion by 2034 (compound annual growth rate \(\sim\)9.0\% for 2024--2034).\cite{globenewswire2025cochlear} Demand is driven by aging populations, persistent noise exposure, and recognition of genetic contributions to hearing loss.\cite{WHO2025} Despite demonstrated benefits, the path to cochlear implantation still presents high barriers: total procedure costs often range from \$50{,}000 to \$100{,}000 when all components of care are included, and access can be constrained by insurance coverage, referral patterns, and center availability. The sector includes established manufacturers such as Cochlear Limited (Australia), Advanced Bionics (Sonova Holding AG, Switzerland), Zhejiang Nurotron Biotechnology Co., Ltd. (China), and MED-EL Medical Electronics (Austria). Regulatory requirements for safety and effectiveness---particularly for novel active implantable devices---and the high cost of research and development contribute to significant barriers to entry, while simultaneously creating opportunities for differentiated, high-value innovation.

\subsection*{2.4 Motivation for higher-fidelity, regenerative interfaces}
Taken together, the epidemiologic scale and the economic burden motivate solutions that improve the benefit-to-cost ratio of hearing care. In the context of cochlear implants, narrowing the effective electrode--neuron distance and increasing the number of addressable neural elements are plausible pathways to improved speech-in-noise understanding, music perception, and spatial hearing. These functional gains would be expected to translate into measurable quality-of-life improvements and economic benefits. This review therefore proceeds from population-level need to the anatomical and engineering questions required to enable ``living'' cochlear implants that cooperate with tissue to improve coupling over time.
