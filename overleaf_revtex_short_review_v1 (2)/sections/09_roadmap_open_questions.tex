\section*{8. Roadmap and open questions}

\subsection*{7.1 Near-term (1--3 years): anatomy, dosing, and proof of integration}

The immediate priority is to confirm that the \CPS\ can be engaged from the \ST\ and that localized cues bias neurite growth toward the modiolus. In human temporal bone and a size‐relevant animal model, \textit{[CPS]} ex vivo microinjections (proteins, dyes, nanoparticles) should map fluid connectivity from medial \ST\ ports into \CPS\ and modiolar spaces with micro‐CT and histology, extending classical demonstrations of perilymph–modiolar communication.\cite{raskandersen2006, sando1971, masuda1971, lim1970} In parallel, \textit{[CPS]} transport modeling anchored to measured resistivities and geometries will propose dose volumes, pulse timing, and port spacing that produce steep, confined gradients at channel inlets.\cite{Micco2006, nella2023} 

With delivery geometry in hand, cochlea‐mimetic assays should evaluate \textit{[CPS]} guidance efficacy: neurite trajectory fidelity and growth fraction toward \CPS‐aligned interfaces under combinations of trophic/gene/vesicle cues, mechanical topographies, and electrical stimulation.\cite{Kempfle2021, StPeter2022, Chang2020, Scheper2019, tan2012, CarnicerLombarte2024AdvMat} On the device side, \textit{[General]} conductive or zwitterionic hydrogel layers should be qualified for chronic stimulation and reduced foreign‐body response, screening insertion mechanics in fresh‐frozen temporal bone.\cite{Dalrymple2020, Horne2023, Rebscher2008, Sheykholeslami2002} 

The near‐term milestone is a preclinical dossier showing (i) anatomical access and gradient control, (ii) neurite bias toward the modiolus in relevant preparations, and (iii) no worsening of insertion trauma relative to standard arrays. Functional readouts at this stage can focus on \textit{[CPS]} interface‐level measures in animals (e.g., ECAP/eABR thresholds, spread of excitation, and channel interactions), reserving behavioral outcomes for later.\cite{wilson2008, wilson2014}

\subsection*{7.2 Mid-term (3--7 years): large-animal safety/efficacy and surgical workflow}

The mid‐term work transitions to chronic, powered implants in a large‐animal model. \textit{[CPS]} Studies should track local tissue response, guided neurite stability, and interface metrics over months of clinical‐like use (duty cycles, stimulation patterns), with serial impedance/SOE mapping and histology at defined endpoints.\cite{Dalrymple2020, Horne2023} \textit{[General]} Manufacturing must lock in reproducible coatings/ports, sterilization compatibility, and packaging; human‐factor evaluations should verify that surgical steps (round‐window or cochleostomy access, array alignment to the medial wall) fit existing workflows.\cite{Rebscher2008} 

Regulatory preparation proceeds in parallel. \textit{[General]} Plan bench and \emph{in vivo} preclinical studies against applicable active‐implantable and CI‐specific requirements;\cite{ISO14708} where modeling supports safety or dosing arguments, document verification/validation to strengthen submissions.\cite{USFDA2021InSilico} If pharmacologic or genetic payloads are used, define maximal safe local exposures and stopping rules. Mid‐term success is a package sufficient for an investigational device/combination‐product submission for a limited FIH feasibility study.

\subsection*{7.3 Long-term (>7 years): clinical integration and scale}

Assuming feasibility is shown, the long‐term agenda is to convert interface‐level gains into clinically meaningful benefits. \textit{[CPS]} Early human studies should target well‐imaged adults (post‐lingual) with patent basal turns, tracking ECAP thresholds, SOE width, and channel independence longitudinally, with exploratory speech‐in‐noise and music metrics to estimate effect sizes.\cite{wilson2008, wilson2014} \textit{[General]} Scaling requires process controls for the biohybrid layer, supply chains for any therapeutic payloads, and compatibility with upgrade paths in commercial platforms. Over time, design space can widen to pediatric indications if safety and benefit hold, with careful attention to growth, plasticity, and long‐term stewardship.

\subsection*{7.4 Open questions}

\textit{[CPS]} \textbf{How frequent and patent are \CPS\ in adult candidate populations?} Classical studies show routes between perilymph and modiolar spaces, but the distribution, diameters, and endosteal coverage relevant to dosing are incompletely characterized in modern cohorts.\cite{raskandersen2006, sando1971, masuda1971} 

\textit{[CPS]} \textbf{What cue combinations minimize off‐target growth while maximizing coupling?} The cochlea’s constraints favor short, steep gradients and multimodal guidance; the optimal balance of trophic, mechanical, and electrical cues remains to be mapped across time.\cite{Kempfle2021, StPeter2022, CarnicerLombarte2024AdvMat}

\textit{[CPS]/[General]} \textbf{What effect sizes at the interface predict perceptual benefit?} Reductions in SOE width and channel interactions plausibly translate to better spectral and temporal resolution, but quantitative links to speech‐in‐noise or music outcomes (and the timescales on which they emerge) need prospective study.\cite{Micco2006, wilson2014}

\textit{[General]} \textbf{How durable are biohybrid materials in the cochlear milieu?} Coatings must withstand micromotion, stimulation, and fluid chemistry for years without delamination, fouling, or impedance drift; further data on long‐term stability in the human basal turn will be determinative.\cite{Dalrymple2020, Horne2023}

\textit{[General]} \textbf{What are the pragmatic regulatory paths for combination products in this space?} Harmonizing device and pharmacologic requirements, right‐sizing preclinical packages, and codifying modeling expectations will govern timelines.\cite{ISO14708, USFDA2021InSilico}

Finally, \textit{[CPS]/[General]} the field needs consensus benchmarks and shared datasets (anatomy, transport, and interface metrics) to enable reproducible comparisons across platforms and cohorts, accelerating translation from proof‐of‐integration to durable patient benefit.\cite{Vecchi2024}
